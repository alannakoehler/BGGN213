\documentclass[]{article}
\usepackage{lmodern}
\usepackage{amssymb,amsmath}
\usepackage{ifxetex,ifluatex}
\usepackage{fixltx2e} % provides \textsubscript
\ifnum 0\ifxetex 1\fi\ifluatex 1\fi=0 % if pdftex
  \usepackage[T1]{fontenc}
  \usepackage[utf8]{inputenc}
\else % if luatex or xelatex
  \ifxetex
    \usepackage{mathspec}
  \else
    \usepackage{fontspec}
  \fi
  \defaultfontfeatures{Ligatures=TeX,Scale=MatchLowercase}
\fi
% use upquote if available, for straight quotes in verbatim environments
\IfFileExists{upquote.sty}{\usepackage{upquote}}{}
% use microtype if available
\IfFileExists{microtype.sty}{%
\usepackage{microtype}
\UseMicrotypeSet[protrusion]{basicmath} % disable protrusion for tt fonts
}{}
\usepackage[margin=1in]{geometry}
\usepackage{hyperref}
\hypersetup{unicode=true,
            pdftitle={Class 8 Machine Learning 1},
            pdfauthor={Alanna Koehler},
            pdfborder={0 0 0},
            breaklinks=true}
\urlstyle{same}  % don't use monospace font for urls
\usepackage{color}
\usepackage{fancyvrb}
\newcommand{\VerbBar}{|}
\newcommand{\VERB}{\Verb[commandchars=\\\{\}]}
\DefineVerbatimEnvironment{Highlighting}{Verbatim}{commandchars=\\\{\}}
% Add ',fontsize=\small' for more characters per line
\usepackage{framed}
\definecolor{shadecolor}{RGB}{248,248,248}
\newenvironment{Shaded}{\begin{snugshade}}{\end{snugshade}}
\newcommand{\AlertTok}[1]{\textcolor[rgb]{0.94,0.16,0.16}{#1}}
\newcommand{\AnnotationTok}[1]{\textcolor[rgb]{0.56,0.35,0.01}{\textbf{\textit{#1}}}}
\newcommand{\AttributeTok}[1]{\textcolor[rgb]{0.77,0.63,0.00}{#1}}
\newcommand{\BaseNTok}[1]{\textcolor[rgb]{0.00,0.00,0.81}{#1}}
\newcommand{\BuiltInTok}[1]{#1}
\newcommand{\CharTok}[1]{\textcolor[rgb]{0.31,0.60,0.02}{#1}}
\newcommand{\CommentTok}[1]{\textcolor[rgb]{0.56,0.35,0.01}{\textit{#1}}}
\newcommand{\CommentVarTok}[1]{\textcolor[rgb]{0.56,0.35,0.01}{\textbf{\textit{#1}}}}
\newcommand{\ConstantTok}[1]{\textcolor[rgb]{0.00,0.00,0.00}{#1}}
\newcommand{\ControlFlowTok}[1]{\textcolor[rgb]{0.13,0.29,0.53}{\textbf{#1}}}
\newcommand{\DataTypeTok}[1]{\textcolor[rgb]{0.13,0.29,0.53}{#1}}
\newcommand{\DecValTok}[1]{\textcolor[rgb]{0.00,0.00,0.81}{#1}}
\newcommand{\DocumentationTok}[1]{\textcolor[rgb]{0.56,0.35,0.01}{\textbf{\textit{#1}}}}
\newcommand{\ErrorTok}[1]{\textcolor[rgb]{0.64,0.00,0.00}{\textbf{#1}}}
\newcommand{\ExtensionTok}[1]{#1}
\newcommand{\FloatTok}[1]{\textcolor[rgb]{0.00,0.00,0.81}{#1}}
\newcommand{\FunctionTok}[1]{\textcolor[rgb]{0.00,0.00,0.00}{#1}}
\newcommand{\ImportTok}[1]{#1}
\newcommand{\InformationTok}[1]{\textcolor[rgb]{0.56,0.35,0.01}{\textbf{\textit{#1}}}}
\newcommand{\KeywordTok}[1]{\textcolor[rgb]{0.13,0.29,0.53}{\textbf{#1}}}
\newcommand{\NormalTok}[1]{#1}
\newcommand{\OperatorTok}[1]{\textcolor[rgb]{0.81,0.36,0.00}{\textbf{#1}}}
\newcommand{\OtherTok}[1]{\textcolor[rgb]{0.56,0.35,0.01}{#1}}
\newcommand{\PreprocessorTok}[1]{\textcolor[rgb]{0.56,0.35,0.01}{\textit{#1}}}
\newcommand{\RegionMarkerTok}[1]{#1}
\newcommand{\SpecialCharTok}[1]{\textcolor[rgb]{0.00,0.00,0.00}{#1}}
\newcommand{\SpecialStringTok}[1]{\textcolor[rgb]{0.31,0.60,0.02}{#1}}
\newcommand{\StringTok}[1]{\textcolor[rgb]{0.31,0.60,0.02}{#1}}
\newcommand{\VariableTok}[1]{\textcolor[rgb]{0.00,0.00,0.00}{#1}}
\newcommand{\VerbatimStringTok}[1]{\textcolor[rgb]{0.31,0.60,0.02}{#1}}
\newcommand{\WarningTok}[1]{\textcolor[rgb]{0.56,0.35,0.01}{\textbf{\textit{#1}}}}
\usepackage{graphicx,grffile}
\makeatletter
\def\maxwidth{\ifdim\Gin@nat@width>\linewidth\linewidth\else\Gin@nat@width\fi}
\def\maxheight{\ifdim\Gin@nat@height>\textheight\textheight\else\Gin@nat@height\fi}
\makeatother
% Scale images if necessary, so that they will not overflow the page
% margins by default, and it is still possible to overwrite the defaults
% using explicit options in \includegraphics[width, height, ...]{}
\setkeys{Gin}{width=\maxwidth,height=\maxheight,keepaspectratio}
\IfFileExists{parskip.sty}{%
\usepackage{parskip}
}{% else
\setlength{\parindent}{0pt}
\setlength{\parskip}{6pt plus 2pt minus 1pt}
}
\setlength{\emergencystretch}{3em}  % prevent overfull lines
\providecommand{\tightlist}{%
  \setlength{\itemsep}{0pt}\setlength{\parskip}{0pt}}
\setcounter{secnumdepth}{0}
% Redefines (sub)paragraphs to behave more like sections
\ifx\paragraph\undefined\else
\let\oldparagraph\paragraph
\renewcommand{\paragraph}[1]{\oldparagraph{#1}\mbox{}}
\fi
\ifx\subparagraph\undefined\else
\let\oldsubparagraph\subparagraph
\renewcommand{\subparagraph}[1]{\oldsubparagraph{#1}\mbox{}}
\fi

%%% Use protect on footnotes to avoid problems with footnotes in titles
\let\rmarkdownfootnote\footnote%
\def\footnote{\protect\rmarkdownfootnote}

%%% Change title format to be more compact
\usepackage{titling}

% Create subtitle command for use in maketitle
\providecommand{\subtitle}[1]{
  \posttitle{
    \begin{center}\large#1\end{center}
    }
}

\setlength{\droptitle}{-2em}

  \title{Class 8 Machine Learning 1}
    \pretitle{\vspace{\droptitle}\centering\huge}
  \posttitle{\par}
    \author{Alanna Koehler}
    \preauthor{\centering\large\emph}
  \postauthor{\par}
      \predate{\centering\large\emph}
  \postdate{\par}
    \date{10/25/2019}


\begin{document}
\maketitle

\#\#K-means example

\#Generate a simple dataset for clustering:

\begin{Shaded}
\begin{Highlighting}[]
\NormalTok{tmp <-}\StringTok{ }\KeywordTok{c}\NormalTok{(}\KeywordTok{rnorm}\NormalTok{(}\DecValTok{30}\NormalTok{,}\OperatorTok{-}\DecValTok{3}\NormalTok{), }\KeywordTok{rnorm}\NormalTok{(}\DecValTok{30}\NormalTok{,}\DecValTok{3}\NormalTok{)) }\CommentTok{#Give you 30 points centered at -3 and 30 points centered at 3.}
\NormalTok{x <-}\StringTok{ }\KeywordTok{cbind}\NormalTok{(}\DataTypeTok{x=}\NormalTok{tmp, }\DataTypeTok{y=}\KeywordTok{rev}\NormalTok{(tmp)) }\CommentTok{#column-wise binds our vector into 2 columns}
\KeywordTok{plot}\NormalTok{(x)}
\end{Highlighting}
\end{Shaded}

\includegraphics{Class08_MachineLearning_files/figure-latex/unnamed-chunk-1-1.pdf}

\#Use the kmeans() function setting k to 2 and nstart=20

\begin{Shaded}
\begin{Highlighting}[]
\NormalTok{k <-}\StringTok{ }\KeywordTok{kmeans}\NormalTok{(x, }\DataTypeTok{centers =} \DecValTok{2}\NormalTok{, }\DataTypeTok{nstart =} \DecValTok{20}\NormalTok{)}
\end{Highlighting}
\end{Shaded}

This gives us a list of 9 elements in our global environment. Print the
results

\begin{Shaded}
\begin{Highlighting}[]
\NormalTok{k}
\end{Highlighting}
\end{Shaded}

\begin{verbatim}
## K-means clustering with 2 clusters of sizes 30, 30
## 
## Cluster means:
##           x         y
## 1  3.002861 -2.489079
## 2 -2.489079  3.002861
## 
## Clustering vector:
##  [1] 2 2 2 2 2 2 2 2 2 2 2 2 2 2 2 2 2 2 2 2 2 2 2 2 2 2 2 2 2 2 1 1 1 1 1
## [36] 1 1 1 1 1 1 1 1 1 1 1 1 1 1 1 1 1 1 1 1 1 1 1 1 1
## 
## Within cluster sum of squares by cluster:
## [1] 62.49912 62.49912
##  (between_SS / total_SS =  87.9 %)
## 
## Available components:
## 
## [1] "cluster"      "centers"      "totss"        "withinss"    
## [5] "tot.withinss" "betweenss"    "size"         "iter"        
## [9] "ifault"
\end{verbatim}

The clustering vector tells you in which cluster each element lies.

\#Inspect the results: Q. How many points are in each cluster?

\begin{Shaded}
\begin{Highlighting}[]
\NormalTok{k}\OperatorTok{$}\NormalTok{size}
\end{Highlighting}
\end{Shaded}

\begin{verbatim}
## [1] 30 30
\end{verbatim}

Q. What `component' of your result object details - cluster size?

\begin{Shaded}
\begin{Highlighting}[]
\NormalTok{k}\OperatorTok{$}\NormalTok{cluster}
\end{Highlighting}
\end{Shaded}

\begin{verbatim}
##  [1] 2 2 2 2 2 2 2 2 2 2 2 2 2 2 2 2 2 2 2 2 2 2 2 2 2 2 2 2 2 2 1 1 1 1 1
## [36] 1 1 1 1 1 1 1 1 1 1 1 1 1 1 1 1 1 1 1 1 1 1 1 1 1
\end{verbatim}

\begin{itemize}
\tightlist
\item
  cluster assignment/membership?
\end{itemize}

\begin{Shaded}
\begin{Highlighting}[]
\KeywordTok{table}\NormalTok{(k}\OperatorTok{$}\NormalTok{cluster)}
\end{Highlighting}
\end{Shaded}

\begin{verbatim}
## 
##  1  2 
## 30 30
\end{verbatim}

\begin{itemize}
\tightlist
\item
  cluster center?
\end{itemize}

\begin{Shaded}
\begin{Highlighting}[]
\NormalTok{k}\OperatorTok{$}\NormalTok{centers}
\end{Highlighting}
\end{Shaded}

\begin{verbatim}
##           x         y
## 1  3.002861 -2.489079
## 2 -2.489079  3.002861
\end{verbatim}

Plot x colored by the kmeans cluster assignment and add cluster centers
as blue points

\begin{Shaded}
\begin{Highlighting}[]
\KeywordTok{plot}\NormalTok{(x, }\DataTypeTok{col =}\NormalTok{ k}\OperatorTok{$}\NormalTok{cluster)}
\KeywordTok{points}\NormalTok{(k}\OperatorTok{$}\NormalTok{centers, }\DataTypeTok{col =} \StringTok{"blue"}\NormalTok{, }\DataTypeTok{pch =} \DecValTok{15}\NormalTok{)}
\end{Highlighting}
\end{Shaded}

\includegraphics{Class08_MachineLearning_files/figure-latex/unnamed-chunk-8-1.pdf}

\#\#Hierarchical clustering example \#First we need to calculate point
(dis)similarity as the Euclidean distance between observations.

\begin{Shaded}
\begin{Highlighting}[]
\NormalTok{dist_matrix <-}\StringTok{ }\KeywordTok{dist}\NormalTok{(x)}
\end{Highlighting}
\end{Shaded}

The hclust() function returns a hierarchical clustering model

\begin{Shaded}
\begin{Highlighting}[]
\NormalTok{hc <-}\StringTok{ }\KeywordTok{hclust}\NormalTok{(}\DataTypeTok{d =}\NormalTok{ dist_matrix)}
\end{Highlighting}
\end{Shaded}

The print method is not so useful here

\begin{Shaded}
\begin{Highlighting}[]
\NormalTok{hc}
\end{Highlighting}
\end{Shaded}

\begin{verbatim}
## 
## Call:
## hclust(d = dist_matrix)
## 
## Cluster method   : complete 
## Distance         : euclidean 
## Number of objects: 60
\end{verbatim}

\#Plot results in a dendrogram

\begin{Shaded}
\begin{Highlighting}[]
\KeywordTok{plot}\NormalTok{(hc)}
\end{Highlighting}
\end{Shaded}

\includegraphics{Class08_MachineLearning_files/figure-latex/unnamed-chunk-12-1.pdf}
\#You then have to cut this tree in a position you like. First you can
generate an abline:

\begin{Shaded}
\begin{Highlighting}[]
\KeywordTok{plot}\NormalTok{(hc)}
\KeywordTok{abline}\NormalTok{(}\DataTypeTok{h=}\DecValTok{6}\NormalTok{, }\DataTypeTok{col=}\StringTok{"red"}\NormalTok{)}
\KeywordTok{abline}\NormalTok{(}\DataTypeTok{h=}\DecValTok{4}\NormalTok{, }\DataTypeTok{col=}\StringTok{"blue"}\NormalTok{)}
\end{Highlighting}
\end{Shaded}

\includegraphics{Class08_MachineLearning_files/figure-latex/unnamed-chunk-13-1.pdf}
\#Cut the tree at a certain height and assign each element to a cluster.

\begin{Shaded}
\begin{Highlighting}[]
\KeywordTok{cutree}\NormalTok{(hc, }\DataTypeTok{h=}\DecValTok{6}\NormalTok{)}
\end{Highlighting}
\end{Shaded}

\begin{verbatim}
##  [1] 1 1 1 1 1 1 1 1 1 1 1 1 1 1 1 1 1 1 1 1 1 1 1 1 1 1 1 1 1 1 2 2 2 2 2
## [36] 2 2 2 2 2 2 2 2 2 2 2 2 2 2 2 2 2 2 2 2 2 2 2 2 2
\end{verbatim}

\begin{Shaded}
\begin{Highlighting}[]
\KeywordTok{cutree}\NormalTok{(hc, }\DataTypeTok{h=}\DecValTok{4}\NormalTok{)}
\end{Highlighting}
\end{Shaded}

\begin{verbatim}
##  [1] 1 1 2 1 2 2 1 1 1 1 1 2 1 2 1 1 1 1 1 2 2 1 1 1 1 1 1 2 2 2 3 3 3 4 4
## [36] 4 4 4 4 3 3 4 4 4 4 4 3 4 3 4 4 4 4 4 3 3 4 3 4 4
\end{verbatim}

You can quantify how many elements are in each cut group by making a
table.

\begin{Shaded}
\begin{Highlighting}[]
\NormalTok{grps <-}\StringTok{ }\KeywordTok{cutree}\NormalTok{(hc, }\DataTypeTok{h=}\DecValTok{6}\NormalTok{)}
\KeywordTok{table}\NormalTok{(grps)}
\end{Highlighting}
\end{Shaded}

\begin{verbatim}
## grps
##  1  2 
## 30 30
\end{verbatim}

Once you have defined the groups, you can then color your clusters based
on those groups.

\begin{Shaded}
\begin{Highlighting}[]
\KeywordTok{plot}\NormalTok{(x, }\DataTypeTok{col =}\NormalTok{ grps)}
\end{Highlighting}
\end{Shaded}

\includegraphics{Class08_MachineLearning_files/figure-latex/unnamed-chunk-16-1.pdf}
You can also cut the tree to yield a given k groups/clusters

\begin{Shaded}
\begin{Highlighting}[]
\KeywordTok{cutree}\NormalTok{(hc, }\DataTypeTok{k=}\DecValTok{2}\NormalTok{)}
\end{Highlighting}
\end{Shaded}

\begin{verbatim}
##  [1] 1 1 1 1 1 1 1 1 1 1 1 1 1 1 1 1 1 1 1 1 1 1 1 1 1 1 1 1 1 1 2 2 2 2 2
## [36] 2 2 2 2 2 2 2 2 2 2 2 2 2 2 2 2 2 2 2 2 2 2 2 2 2
\end{verbatim}

\hypertarget{using-different-hierarchical-clustering-methods}{%
\subsection{Using different hierarchical clustering
methods}\label{using-different-hierarchical-clustering-methods}}

\#Step 1. Generate some example data for clustering

\begin{Shaded}
\begin{Highlighting}[]
\NormalTok{x <-}\StringTok{ }\KeywordTok{rbind}\NormalTok{(}
 \KeywordTok{matrix}\NormalTok{(}\KeywordTok{rnorm}\NormalTok{(}\DecValTok{100}\NormalTok{, }\DataTypeTok{mean=}\DecValTok{0}\NormalTok{, }\DataTypeTok{sd =} \FloatTok{0.3}\NormalTok{), }\DataTypeTok{ncol =} \DecValTok{2}\NormalTok{), }\CommentTok{# c1}
 \KeywordTok{matrix}\NormalTok{(}\KeywordTok{rnorm}\NormalTok{(}\DecValTok{100}\NormalTok{, }\DataTypeTok{mean =} \DecValTok{1}\NormalTok{, }\DataTypeTok{sd =} \FloatTok{0.3}\NormalTok{), }\DataTypeTok{ncol =} \DecValTok{2}\NormalTok{), }\CommentTok{# c2}
 \KeywordTok{matrix}\NormalTok{(}\KeywordTok{c}\NormalTok{(}\KeywordTok{rnorm}\NormalTok{(}\DecValTok{50}\NormalTok{, }\DataTypeTok{mean =} \DecValTok{1}\NormalTok{, }\DataTypeTok{sd =} \FloatTok{0.3}\NormalTok{), }\CommentTok{# c3}
 \KeywordTok{rnorm}\NormalTok{(}\DecValTok{50}\NormalTok{, }\DataTypeTok{mean =} \DecValTok{0}\NormalTok{, }\DataTypeTok{sd =} \FloatTok{0.3}\NormalTok{)), }\DataTypeTok{ncol =} \DecValTok{2}\NormalTok{))}
\KeywordTok{colnames}\NormalTok{(x) <-}\StringTok{ }\KeywordTok{c}\NormalTok{(}\StringTok{"x"}\NormalTok{, }\StringTok{"y"}\NormalTok{)}
\end{Highlighting}
\end{Shaded}

\#Step 2. Plot the data without clustering

\begin{Shaded}
\begin{Highlighting}[]
\KeywordTok{plot}\NormalTok{(x)}
\end{Highlighting}
\end{Shaded}

\includegraphics{Class08_MachineLearning_files/figure-latex/unnamed-chunk-19-1.pdf}
\#Step 3. Generate colors for known clusters (just so we can compare to
hclust results).

\begin{Shaded}
\begin{Highlighting}[]
\NormalTok{col <-}\StringTok{ }\KeywordTok{as.factor}\NormalTok{( }\KeywordTok{rep}\NormalTok{(}\KeywordTok{c}\NormalTok{(}\StringTok{"c1"}\NormalTok{,}\StringTok{"c2"}\NormalTok{,}\StringTok{"c3"}\NormalTok{), }\DataTypeTok{each=}\DecValTok{50}\NormalTok{) )}
\KeywordTok{plot}\NormalTok{(x, }\DataTypeTok{col=}\NormalTok{col)}
\end{Highlighting}
\end{Shaded}

\includegraphics{Class08_MachineLearning_files/figure-latex/unnamed-chunk-20-1.pdf}

\#Answer questions Use the dist(), hclust(), plot() and cutree()
functions to return 2 and 3 clusters. How does this compare to your
known col groups.

\begin{Shaded}
\begin{Highlighting}[]
\NormalTok{hc2 <-}\StringTok{ }\KeywordTok{hclust}\NormalTok{(}\KeywordTok{dist}\NormalTok{(x))}
\KeywordTok{plot}\NormalTok{(hc2)}
\KeywordTok{abline}\NormalTok{(}\DataTypeTok{h=}\FloatTok{2.5}\NormalTok{, }\DataTypeTok{col =} \StringTok{"red"}\NormalTok{)}
\KeywordTok{abline}\NormalTok{(}\DataTypeTok{h=}\DecValTok{2}\NormalTok{, }\DataTypeTok{col =} \StringTok{"blue"}\NormalTok{)}
\end{Highlighting}
\end{Shaded}

\includegraphics{Class08_MachineLearning_files/figure-latex/unnamed-chunk-21-1.pdf}

\begin{Shaded}
\begin{Highlighting}[]
\NormalTok{grps2 <-}\StringTok{ }\KeywordTok{cutree}\NormalTok{(hc2, }\DataTypeTok{k=}\DecValTok{2}\NormalTok{)}
\KeywordTok{table}\NormalTok{(grps2)}
\end{Highlighting}
\end{Shaded}

\begin{verbatim}
## grps2
##  1  2 
## 70 80
\end{verbatim}

\begin{Shaded}
\begin{Highlighting}[]
\KeywordTok{plot}\NormalTok{(x, }\DataTypeTok{col =}\NormalTok{ grps2)}
\end{Highlighting}
\end{Shaded}

\includegraphics{Class08_MachineLearning_files/figure-latex/unnamed-chunk-23-1.pdf}

\begin{Shaded}
\begin{Highlighting}[]
\KeywordTok{table}\NormalTok{(col, grps2)}
\end{Highlighting}
\end{Shaded}

\begin{verbatim}
##     grps2
## col   1  2
##   c1 50  0
##   c2  0 50
##   c3 20 30
\end{verbatim}

\begin{Shaded}
\begin{Highlighting}[]
\NormalTok{grps3 <-}\StringTok{ }\KeywordTok{cutree}\NormalTok{(hc2, }\DataTypeTok{k=}\DecValTok{3}\NormalTok{)}
\KeywordTok{table}\NormalTok{(grps3)}
\end{Highlighting}
\end{Shaded}

\begin{verbatim}
## grps3
##  1  2  3 
## 70 47 33
\end{verbatim}

\begin{Shaded}
\begin{Highlighting}[]
\KeywordTok{plot}\NormalTok{(x, }\DataTypeTok{col =}\NormalTok{ grps3)}
\end{Highlighting}
\end{Shaded}

\includegraphics{Class08_MachineLearning_files/figure-latex/unnamed-chunk-26-1.pdf}

\begin{Shaded}
\begin{Highlighting}[]
\KeywordTok{table}\NormalTok{(col, grps3)}
\end{Highlighting}
\end{Shaded}

\begin{verbatim}
##     grps3
## col   1  2  3
##   c1 50  0  0
##   c2  0 45  5
##   c3 20  2 28
\end{verbatim}

\#\#Principle Component Analysis

\#Plot example data

\begin{Shaded}
\begin{Highlighting}[]
\NormalTok{mydata <-}\StringTok{ }\KeywordTok{read.csv}\NormalTok{(}\StringTok{"https://tinyurl.com/expression-CSV"}\NormalTok{,}
 \DataTypeTok{row.names=}\DecValTok{1}\NormalTok{)}
\KeywordTok{head}\NormalTok{(mydata)}
\end{Highlighting}
\end{Shaded}

\begin{verbatim}
##        wt1 wt2  wt3  wt4 wt5 ko1 ko2 ko3 ko4 ko5
## gene1  439 458  408  429 420  90  88  86  90  93
## gene2  219 200  204  210 187 427 423 434 433 426
## gene3 1006 989 1030 1017 973 252 237 238 226 210
## gene4  783 792  829  856 760 849 856 835 885 894
## gene5  181 249  204  244 225 277 305 272 270 279
## gene6  460 502  491  491 493 612 594 577 618 638
\end{verbatim}

Note that samples are columns and genes are rows.

\#Now define your PCA components and look at the attributes

\begin{Shaded}
\begin{Highlighting}[]
\NormalTok{pca <-}\StringTok{ }\KeywordTok{prcomp}\NormalTok{(}\KeywordTok{t}\NormalTok{(mydata), }\DataTypeTok{scale=}\OtherTok{TRUE}\NormalTok{) }\CommentTok{#Need to transpose with t() bc prcomp expects samples to be rows and genes to be columns}
\KeywordTok{attributes}\NormalTok{(pca) }
\end{Highlighting}
\end{Shaded}

\begin{verbatim}
## $names
## [1] "sdev"     "rotation" "center"   "scale"    "x"       
## 
## $class
## [1] "prcomp"
\end{verbatim}

\#Make a basic PC1 vs PC2 plot.

\begin{Shaded}
\begin{Highlighting}[]
\KeywordTok{plot}\NormalTok{(pca}\OperatorTok{$}\NormalTok{x[,}\DecValTok{1}\NormalTok{], pca}\OperatorTok{$}\NormalTok{x[,}\DecValTok{2}\NormalTok{]) }
\end{Highlighting}
\end{Shaded}

\includegraphics{Class08_MachineLearning_files/figure-latex/unnamed-chunk-30-1.pdf}
\#Calculate variance captured by each PC.

\begin{Shaded}
\begin{Highlighting}[]
\NormalTok{pca.var <-}\StringTok{ }\NormalTok{pca}\OperatorTok{$}\NormalTok{sdev}\OperatorTok{^}\DecValTok{2} 
\NormalTok{pca.var}
\end{Highlighting}
\end{Shaded}

\begin{verbatim}
##  [1] 9.261625e+01 2.309940e+00 1.119081e+00 1.106773e+00 7.754851e-01
##  [6] 6.813705e-01 6.417798e-01 3.852003e-01 3.641185e-01 1.121154e-29
\end{verbatim}

You can also look at percent variance.

\begin{Shaded}
\begin{Highlighting}[]
\NormalTok{pca.var <-}\StringTok{ }\NormalTok{pca}\OperatorTok{$}\NormalTok{sdev}\OperatorTok{^}\DecValTok{2}
\NormalTok{pca.var.per <-}\StringTok{ }\KeywordTok{round}\NormalTok{(pca.var}\OperatorTok{/}\KeywordTok{sum}\NormalTok{(pca.var)}\OperatorTok{*}\DecValTok{100}\NormalTok{, }\DecValTok{1}\NormalTok{) }
\NormalTok{pca.var.per}
\end{Highlighting}
\end{Shaded}

\begin{verbatim}
##  [1] 92.6  2.3  1.1  1.1  0.8  0.7  0.6  0.4  0.4  0.0
\end{verbatim}

Make a scree plot.

\begin{Shaded}
\begin{Highlighting}[]
\KeywordTok{barplot}\NormalTok{(pca.var.per, }\DataTypeTok{main=}\StringTok{"Scree Plot"}\NormalTok{,}
 \DataTypeTok{xlab=}\StringTok{"Principal Component"}\NormalTok{, }\DataTypeTok{ylab=}\StringTok{"Percent Variation"}\NormalTok{)}
\end{Highlighting}
\end{Shaded}

\includegraphics{Class08_MachineLearning_files/figure-latex/unnamed-chunk-33-1.pdf}

Add color to your PCA plot.

\begin{Shaded}
\begin{Highlighting}[]
\NormalTok{colvec <-}\StringTok{ }\KeywordTok{as.factor}\NormalTok{( }\KeywordTok{substr}\NormalTok{( }\KeywordTok{colnames}\NormalTok{(mydata), }\DecValTok{1}\NormalTok{, }\DecValTok{2}\NormalTok{) )}
\KeywordTok{plot}\NormalTok{(pca}\OperatorTok{$}\NormalTok{x[,}\DecValTok{1}\NormalTok{], pca}\OperatorTok{$}\NormalTok{x[,}\DecValTok{2}\NormalTok{], }\DataTypeTok{col=}\NormalTok{colvec, }\DataTypeTok{pch=}\DecValTok{16}\NormalTok{,}
 \DataTypeTok{xlab=}\KeywordTok{paste0}\NormalTok{(}\StringTok{"PC1 ("}\NormalTok{, pca.var.per[}\DecValTok{1}\NormalTok{], }\StringTok{"%)"}\NormalTok{),}
 \DataTypeTok{ylab=}\KeywordTok{paste0}\NormalTok{(}\StringTok{"PC2 ("}\NormalTok{, pca.var.per[}\DecValTok{2}\NormalTok{], }\StringTok{"%)"}\NormalTok{))}
\end{Highlighting}
\end{Shaded}

\includegraphics{Class08_MachineLearning_files/figure-latex/unnamed-chunk-34-1.pdf}


\end{document}
